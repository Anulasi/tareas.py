\documentclass[9pt]{beamer}
% Created By Gouthaman KG
%~~~~~~~~~~~~~~~~~~~~~~~~~~~~~~~~~~~~~~~~~~~~~~~~~~~~~~~~~~~~~~~~~~~~~~~~~~~~~~
% Use roboto Font (recommended)
\usepackage[sfdefault]{roboto}
\usepackage[utf8]{inputenc}
\usepackage[T1]{fontenc}
%~~~~~~~~~~~~~~~~~~~~~~~~~~~~~~~~~~~~~~~~~~~~~~~~~~~~~~~~~~~~~~~~~~~~~~~~~~~~~~

%~~~~~~~~~~~~~~~~~~~~~~~~~~~~~~~~~~~~~~~~~~~~~~~~~~~~~~~~~~~~~~~~~~~~~~~~~~~~~~
% Define where theme files are located. ('/styles')
\usepackage{styles/fluxmacros}
\usefolder{styles}
% Use Flux theme v0.1 beta
% Available style: asphalt, blue, red, green, gray 
\usetheme[style=asphalt]{flux}
%~~~~~~~~~~~~~~~~~~~~~~~~~~~~~~~~~~~~~~~~~~~~~~~~~~~~~~~~~~~~~~~~~~~~~~~~~~~~~~

%~~~~~~~~~~~~~~~~~~~~~~~~~~~~~~~~~~~~~~~~~~~~~~~~~~~~~~~~~~~~~~~~~~~~~~~~~~~~~~
% Extra packages for the demo:
\usepackage{booktabs}
\usepackage{colortbl}
\usepackage{ragged2e}
\usepackage{schemabloc}
\usepackage{hyperref}
\usebackgroundtemplate{
\includegraphics[width=\paperwidth,height=\paperheight]{assets/background.jpg}}%change this to your preferred background for the presentation.
%~~~~~~~~~~~~~~~~~~~~~~~~~~~~~~~~~~~~~~~~~~~~~~~~~~~~~~~~~~~~~~~~~~~~~~~~~~~~~~

%~~~~~~~~~~~~~~~~~~~~~~~~~~~~~~~~~~~~~~~~~~~~~~~~~~~~~~~~~~~~~~~~~~~~~~~~~~~~~~
% Informations
\title{Acoplamiento molecular SARS-CoV proteína E}
\subtitle{}

\author{María Isabel Iñiguez Luna }
\institute{Centro de Investigaciones Cerebrales}
\titlegraphic{} %change this to your preferred logo or image(the image is located on the top right corner).
%~~~~~~~~~~~~~~~~~~~~~~~~~~~~~~~~~~~~~~~~~~~~~~~~~~~~~~~~~~~~~~~~~~~~~~~~~~~~~~

\begin{document}

% Generate title page
\titlepage

\begin{frame}

 \frametitle{Introducción}
 \tableofcontents
\end{frame}

\section{Acoplamiento molecular} %the content in the section will be displayed in the table of contents
\begin{frame}{Docking}%the content in the frame will be displayed as the title of the page
"Los análisis in silico de acoplamiento molecular (AM), permite predecir la estructura más probable entre un ligante y el sitió de unión de una enzima específica, y así establecer una conjetura con base en la energía libre de unión de las conformaciones energéticamente más favorecidas o estables, es decir, aquellas que requieren del menor gasto energético y son más probables a ocurrir.."
    
\end{frame}

\section{SARS-CoV-2}
\begin{frame}{Dianas terapeúticas}
\begin{itemize}

  \setlength\itemsep{1em}  %increase the space between items.
  
    \item Los estudios encaminados a los coronavirus (CoV),han cobrado interés primario en el campo de la investigación para ampliar rápidamente los conocimientos científicos sobre el nuevo virus COVID-19, responsable actual del brote de la enfermedad por coronavirus (OMS,2021) con la intención preservar la salud y prevenir la propagación del brote . 
    
    \item Las principales proteínas blanco en las investigaciones actuales, son la proteína S (pico), y E (envoltura) del coronavirus.
\end{itemize}
\end{frame}

\section{Estrategias preventivas}
\begin{frame}{Estrategias preventivas}
\begin{itemize}

  \setlength\itemsep{1em}  %increase the space between items.
  
    \item Un enfoque prometedor para la prevención, son el diseño de las vacunas vivas atenuadas, y algunas de estas vacunas son dirigidas a la proteína de la envoltura (E), que es una pequeña proteína de membrana que forma canales iónicos (Surya et al., 2018). 
    \end{itemize}
\end{frame}

\section{Planteamiento del problema}
\begin{frame}{Proyecto}
   \begin{itemize}
   \item Identificar la secuencia que codifica a la proteína de envoltura (E) del coronavirus SARS-CoV-2, a partir de la secuencia completa del genoma (NCBI) y evaluar mediante el análisis de acoplamiento molecular (Autodock) si la amantadina es un ligante con potencial farmacológico. 
   
   
   \end{itemize}
 
\end{frame}
    

\end{document}